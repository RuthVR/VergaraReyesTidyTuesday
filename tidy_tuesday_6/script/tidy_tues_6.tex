% Options for packages loaded elsewhere
\PassOptionsToPackage{unicode}{hyperref}
\PassOptionsToPackage{hyphens}{url}
%
\documentclass[
]{article}
\usepackage{lmodern}
\usepackage{amsmath}
\usepackage{ifxetex,ifluatex}
\ifnum 0\ifxetex 1\fi\ifluatex 1\fi=0 % if pdftex
  \usepackage[T1]{fontenc}
  \usepackage[utf8]{inputenc}
  \usepackage{textcomp} % provide euro and other symbols
  \usepackage{amssymb}
\else % if luatex or xetex
  \usepackage{unicode-math}
  \defaultfontfeatures{Scale=MatchLowercase}
  \defaultfontfeatures[\rmfamily]{Ligatures=TeX,Scale=1}
\fi
% Use upquote if available, for straight quotes in verbatim environments
\IfFileExists{upquote.sty}{\usepackage{upquote}}{}
\IfFileExists{microtype.sty}{% use microtype if available
  \usepackage[]{microtype}
  \UseMicrotypeSet[protrusion]{basicmath} % disable protrusion for tt fonts
}{}
\makeatletter
\@ifundefined{KOMAClassName}{% if non-KOMA class
  \IfFileExists{parskip.sty}{%
    \usepackage{parskip}
  }{% else
    \setlength{\parindent}{0pt}
    \setlength{\parskip}{6pt plus 2pt minus 1pt}}
}{% if KOMA class
  \KOMAoptions{parskip=half}}
\makeatother
\usepackage{xcolor}
\IfFileExists{xurl.sty}{\usepackage{xurl}}{} % add URL line breaks if available
\IfFileExists{bookmark.sty}{\usepackage{bookmark}}{\usepackage{hyperref}}
\hypersetup{
  pdftitle={6th Tidy Tuesday},
  pdfauthor={Ruth Vergara Reyes},
  hidelinks,
  pdfcreator={LaTeX via pandoc}}
\urlstyle{same} % disable monospaced font for URLs
\usepackage[margin=1in]{geometry}
\usepackage{color}
\usepackage{fancyvrb}
\newcommand{\VerbBar}{|}
\newcommand{\VERB}{\Verb[commandchars=\\\{\}]}
\DefineVerbatimEnvironment{Highlighting}{Verbatim}{commandchars=\\\{\}}
% Add ',fontsize=\small' for more characters per line
\usepackage{framed}
\definecolor{shadecolor}{RGB}{248,248,248}
\newenvironment{Shaded}{\begin{snugshade}}{\end{snugshade}}
\newcommand{\AlertTok}[1]{\textcolor[rgb]{0.94,0.16,0.16}{#1}}
\newcommand{\AnnotationTok}[1]{\textcolor[rgb]{0.56,0.35,0.01}{\textbf{\textit{#1}}}}
\newcommand{\AttributeTok}[1]{\textcolor[rgb]{0.77,0.63,0.00}{#1}}
\newcommand{\BaseNTok}[1]{\textcolor[rgb]{0.00,0.00,0.81}{#1}}
\newcommand{\BuiltInTok}[1]{#1}
\newcommand{\CharTok}[1]{\textcolor[rgb]{0.31,0.60,0.02}{#1}}
\newcommand{\CommentTok}[1]{\textcolor[rgb]{0.56,0.35,0.01}{\textit{#1}}}
\newcommand{\CommentVarTok}[1]{\textcolor[rgb]{0.56,0.35,0.01}{\textbf{\textit{#1}}}}
\newcommand{\ConstantTok}[1]{\textcolor[rgb]{0.00,0.00,0.00}{#1}}
\newcommand{\ControlFlowTok}[1]{\textcolor[rgb]{0.13,0.29,0.53}{\textbf{#1}}}
\newcommand{\DataTypeTok}[1]{\textcolor[rgb]{0.13,0.29,0.53}{#1}}
\newcommand{\DecValTok}[1]{\textcolor[rgb]{0.00,0.00,0.81}{#1}}
\newcommand{\DocumentationTok}[1]{\textcolor[rgb]{0.56,0.35,0.01}{\textbf{\textit{#1}}}}
\newcommand{\ErrorTok}[1]{\textcolor[rgb]{0.64,0.00,0.00}{\textbf{#1}}}
\newcommand{\ExtensionTok}[1]{#1}
\newcommand{\FloatTok}[1]{\textcolor[rgb]{0.00,0.00,0.81}{#1}}
\newcommand{\FunctionTok}[1]{\textcolor[rgb]{0.00,0.00,0.00}{#1}}
\newcommand{\ImportTok}[1]{#1}
\newcommand{\InformationTok}[1]{\textcolor[rgb]{0.56,0.35,0.01}{\textbf{\textit{#1}}}}
\newcommand{\KeywordTok}[1]{\textcolor[rgb]{0.13,0.29,0.53}{\textbf{#1}}}
\newcommand{\NormalTok}[1]{#1}
\newcommand{\OperatorTok}[1]{\textcolor[rgb]{0.81,0.36,0.00}{\textbf{#1}}}
\newcommand{\OtherTok}[1]{\textcolor[rgb]{0.56,0.35,0.01}{#1}}
\newcommand{\PreprocessorTok}[1]{\textcolor[rgb]{0.56,0.35,0.01}{\textit{#1}}}
\newcommand{\RegionMarkerTok}[1]{#1}
\newcommand{\SpecialCharTok}[1]{\textcolor[rgb]{0.00,0.00,0.00}{#1}}
\newcommand{\SpecialStringTok}[1]{\textcolor[rgb]{0.31,0.60,0.02}{#1}}
\newcommand{\StringTok}[1]{\textcolor[rgb]{0.31,0.60,0.02}{#1}}
\newcommand{\VariableTok}[1]{\textcolor[rgb]{0.00,0.00,0.00}{#1}}
\newcommand{\VerbatimStringTok}[1]{\textcolor[rgb]{0.31,0.60,0.02}{#1}}
\newcommand{\WarningTok}[1]{\textcolor[rgb]{0.56,0.35,0.01}{\textbf{\textit{#1}}}}
\usepackage{graphicx}
\makeatletter
\def\maxwidth{\ifdim\Gin@nat@width>\linewidth\linewidth\else\Gin@nat@width\fi}
\def\maxheight{\ifdim\Gin@nat@height>\textheight\textheight\else\Gin@nat@height\fi}
\makeatother
% Scale images if necessary, so that they will not overflow the page
% margins by default, and it is still possible to overwrite the defaults
% using explicit options in \includegraphics[width, height, ...]{}
\setkeys{Gin}{width=\maxwidth,height=\maxheight,keepaspectratio}
% Set default figure placement to htbp
\makeatletter
\def\fps@figure{htbp}
\makeatother
\setlength{\emergencystretch}{3em} % prevent overfull lines
\providecommand{\tightlist}{%
  \setlength{\itemsep}{0pt}\setlength{\parskip}{0pt}}
\setcounter{secnumdepth}{-\maxdimen} % remove section numbering
\ifluatex
  \usepackage{selnolig}  % disable illegal ligatures
\fi

\title{6th Tidy Tuesday}
\author{Ruth Vergara Reyes}
\date{4/1/2021}

\begin{document}
\maketitle

{
\setcounter{tocdepth}{2}
\tableofcontents
}
\hypertarget{introduction}{%
\section{Introduction}\label{introduction}}

This is the 6th tidy tuesday I make

\hypertarget{load-libraries}{%
\section{Load Libraries}\label{load-libraries}}

\begin{Shaded}
\begin{Highlighting}[]
\FunctionTok{library}\NormalTok{(tidyverse)}
\FunctionTok{library}\NormalTok{(here)}
\FunctionTok{library}\NormalTok{(RColorBrewer)}
\FunctionTok{library}\NormalTok{(tidytuesdayR)}
\FunctionTok{library}\NormalTok{(patchwork)}
\end{Highlighting}
\end{Shaded}

\hypertarget{load-data}{%
\section{Load Data}\label{load-data}}

\begin{Shaded}
\begin{Highlighting}[]
\NormalTok{tues\_data }\OtherTok{\textless{}{-}}\NormalTok{ tidytuesdayR}\SpecialCharTok{::}\FunctionTok{tt\_load}\NormalTok{(}\DecValTok{2021}\NormalTok{, }\AttributeTok{week =} \DecValTok{14}\NormalTok{)}
\end{Highlighting}
\end{Shaded}

\begin{verbatim}
## 
##  Downloading file 1 of 5: `ulta.csv`
##  Downloading file 2 of 5: `sephora.csv`
##  Downloading file 3 of 5: `allShades.csv`
##  Downloading file 4 of 5: `allNumbers.csv`
##  Downloading file 5 of 5: `allCategories.csv`
\end{verbatim}

\begin{Shaded}
\begin{Highlighting}[]
\NormalTok{ulta }\OtherTok{\textless{}{-}}\NormalTok{ tues\_data}\SpecialCharTok{$}\NormalTok{ulta}
\NormalTok{sephora }\OtherTok{\textless{}{-}}\NormalTok{ tues\_data}\SpecialCharTok{$}\NormalTok{sephora}
\NormalTok{allShades }\OtherTok{\textless{}{-}}\NormalTok{ tues\_data}\SpecialCharTok{$}\NormalTok{allShades}
\NormalTok{allNumbers }\OtherTok{\textless{}{-}}\NormalTok{ tues\_data}\SpecialCharTok{$}\NormalTok{allNumbers}
\NormalTok{allCategories }\OtherTok{\textless{}{-}}\NormalTok{ tues\_data}\SpecialCharTok{$}\NormalTok{allCategories}
\end{Highlighting}
\end{Shaded}

\hypertarget{data-analysis}{%
\section{Data Analysis}\label{data-analysis}}

I will be looking at the color range of foundation available in Sephora
and Ulta. I will be comparing ranges avalible at the different places
and then one brand they have that is supposed to have wider ranges.

\hypertarget{cleaning-up-the-data}{%
\subsubsection{Cleaning up the data}\label{cleaning-up-the-data}}

\begin{Shaded}
\begin{Highlighting}[]
\NormalTok{top\_brands\_sephora }\OtherTok{\textless{}{-}} \FunctionTok{c}\NormalTok{(}\StringTok{"Anastasia Beverly Hills"}\NormalTok{, }\StringTok{"bareMinerals"}\NormalTok{, }\StringTok{"Lancôme"}\NormalTok{, }\StringTok{"FENTY BEAUTY by Rihanna"}\NormalTok{)  }\DocumentationTok{\#\#\# Three brands that are the same as Ulta and one that is supposed to have a wider range. \#}

\NormalTok{top\_brands\_ulta }\OtherTok{\textless{}{-}} \FunctionTok{c}\NormalTok{(}\StringTok{"Anastasia Beverly Hills"}\NormalTok{,}\StringTok{"bareMinerals"}\NormalTok{,}\StringTok{"Lancôme"}\NormalTok{, }\StringTok{"Maybelline"}\NormalTok{)                    }\DocumentationTok{\#\#\# Three brands that are the same as Sephora and one that is supposed to have a wider range. \#}

\NormalTok{clean\_sephora }\OtherTok{\textless{}{-}}\NormalTok{ sephora }\SpecialCharTok{\%\textgreater{}\%}
  \FunctionTok{filter}\NormalTok{ (brand }\SpecialCharTok{\%in\%}\NormalTok{ top\_brands\_sephora) }\SpecialCharTok{\%\textgreater{}\%}            \DocumentationTok{\#\#\# How I got the brands I wanted}
  \FunctionTok{arrange}\NormalTok{(brand) }\SpecialCharTok{\%\textgreater{}\%}                                    \DocumentationTok{\#\#\# How I got the brands to be in alphabetical order}
  \FunctionTok{select}\NormalTok{(}\SpecialCharTok{{-}}\NormalTok{url, }\SpecialCharTok{{-}}\NormalTok{imgSrc, }\SpecialCharTok{{-}}\NormalTok{imgAlt, }\SpecialCharTok{{-}}\NormalTok{name)                 }\DocumentationTok{\#\#\# How I got rid of columns that I did not need}

\NormalTok{clean\_ulta }\OtherTok{\textless{}{-}}\NormalTok{ ulta }\SpecialCharTok{\%\textgreater{}\%}
  \FunctionTok{filter}\NormalTok{(brand }\SpecialCharTok{\%in\%}\NormalTok{ top\_brands\_ulta) }\SpecialCharTok{\%\textgreater{}\%}                \DocumentationTok{\#\#\# How I got the brands I wanted to look at }
  \FunctionTok{select}\NormalTok{(}\SpecialCharTok{{-}}\NormalTok{url, }\SpecialCharTok{{-}}\NormalTok{imgSrc, }\SpecialCharTok{{-}}\NormalTok{imgAlt,}\SpecialCharTok{{-}}\NormalTok{name)}\SpecialCharTok{\%\textgreater{}\%}               \DocumentationTok{\#\#\# How I got rid of columns that I did not need}
  \FunctionTok{arrange}\NormalTok{(brand)                                        }\DocumentationTok{\#\#\# How I got the brands to get into alphabetical order}

\NormalTok{clean\_seph\_num }\OtherTok{\textless{}{-}}\NormalTok{ allNumbers }\SpecialCharTok{\%\textgreater{}\%}                        \DocumentationTok{\#\#\# This is how I cleaned the numbers data so I can join with precious data}
  \FunctionTok{filter}\NormalTok{(brand }\SpecialCharTok{\%in\%}\NormalTok{ top\_brands\_sephora)}\SpecialCharTok{\%\textgreater{}\%}              \DocumentationTok{\#\#\# How I got the information that I needed from the numbers data}
  \FunctionTok{select}\NormalTok{(}\SpecialCharTok{{-}}\NormalTok{name, }\SpecialCharTok{{-}}\NormalTok{hex)                                   }\DocumentationTok{\#\#\# How I got rid of columns that I do not need}

\NormalTok{Seph\_data }\OtherTok{\textless{}{-}} \FunctionTok{inner\_join}\NormalTok{(clean\_seph\_num, clean\_sephora)  }\DocumentationTok{\#\#\# How I joined the two sets of data for Sephora into one big data set}

\NormalTok{clean\_ulta\_num }\OtherTok{\textless{}{-}}\NormalTok{ allNumbers }\SpecialCharTok{\%\textgreater{}\%}                        \DocumentationTok{\#\#\# This is the data set I will be joining with the previous ULta data set}
  \FunctionTok{filter}\NormalTok{(brand }\SpecialCharTok{\%in\%}\NormalTok{ top\_brands\_ulta)}\SpecialCharTok{\%\textgreater{}\%}                 \DocumentationTok{\#\#\# How I cleaned up the numbers data for the ulta brands}
  \FunctionTok{select}\NormalTok{(}\SpecialCharTok{{-}}\NormalTok{name, }\SpecialCharTok{{-}}\NormalTok{hex)                                   }\DocumentationTok{\#\#\# How I got rid of columns that I do not need}

\NormalTok{ulta\_data }\OtherTok{\textless{}{-}} \FunctionTok{inner\_join}\NormalTok{(clean\_ulta\_num, clean\_ulta)     }\DocumentationTok{\#\#\# How I joined the two data sets for Ulta into one }
\end{Highlighting}
\end{Shaded}

\hypertarget{making-the-plots}{%
\subsubsection{Making the Plots}\label{making-the-plots}}

I will first make two plots then combine them into one.

\begin{Shaded}
\begin{Highlighting}[]
\DocumentationTok{\#\#\#\#\# Sephora Plot \#\#\#\#\#}

\NormalTok{seph\_plot }\OtherTok{\textless{}{-}}\NormalTok{ Seph\_data }\SpecialCharTok{\%\textgreater{}\%}
  \FunctionTok{ggplot}\NormalTok{(}\FunctionTok{aes}\NormalTok{(}\AttributeTok{x =}\NormalTok{ numbers,             }
             \AttributeTok{y =}\NormalTok{ lightness,}
             \AttributeTok{color =}\NormalTok{ lightness))}\SpecialCharTok{+}               \DocumentationTok{\#\#\# What I chose for my x/y axis}
  \FunctionTok{geom\_point}\NormalTok{()}\SpecialCharTok{+}                                 \DocumentationTok{\#\#\# Type of plot I used}
  \FunctionTok{facet\_wrap}\NormalTok{(}\SpecialCharTok{\textasciitilde{}}\NormalTok{brand, }\AttributeTok{scales =} \StringTok{"free"}\NormalTok{)}\SpecialCharTok{+}          \DocumentationTok{\#\#\# How I got each brand to have its own small plot}
  \FunctionTok{theme\_classic}\NormalTok{()}\SpecialCharTok{+}                              \DocumentationTok{\#\#\# Theme }
  \FunctionTok{labs}\NormalTok{(}\AttributeTok{x =} \StringTok{"Foundation Number"}\NormalTok{,}
       \AttributeTok{y =} \StringTok{"Shade Range "}\NormalTok{,}
       \AttributeTok{title =} \StringTok{"Foundation Range by Popular Brands"}\NormalTok{,}
       \AttributeTok{subtitle =} \StringTok{"This plot shows the shade ranges }
\StringTok{of foundation avaliable at Sephora based on the brand."}\NormalTok{)}\SpecialCharTok{+} \DocumentationTok{\#\#\# new labels for the plot \#\#}
  \FunctionTok{theme}\NormalTok{(}\AttributeTok{legend.box =} \StringTok{"none"}\NormalTok{,}
        \AttributeTok{axis.title =} \FunctionTok{element\_text}\NormalTok{(}\AttributeTok{size =} \DecValTok{10}\NormalTok{),}
        \AttributeTok{title =} \FunctionTok{element\_text}\NormalTok{(}\AttributeTok{size =} \DecValTok{16}\NormalTok{),}
        \AttributeTok{plot.subtitle =} \FunctionTok{element\_text}\NormalTok{(}\AttributeTok{size =} \DecValTok{11}\NormalTok{),}
        \AttributeTok{legend.position =} \StringTok{"none"}\NormalTok{)}\SpecialCharTok{+}              \DocumentationTok{\#\#\# How I changed certain aspects of the plot}
 \FunctionTok{ggsave}\NormalTok{(}\FunctionTok{here}\NormalTok{(}\StringTok{"tidy\_tuesday\_6"}\NormalTok{, }\StringTok{"output"}\NormalTok{, }\StringTok{"seph\_plot.png"}\NormalTok{),}
       \AttributeTok{width =} \DecValTok{10}\NormalTok{, }\AttributeTok{height =} \DecValTok{7}\NormalTok{)}
\NormalTok{seph\_plot}
\end{Highlighting}
\end{Shaded}

\includegraphics{tidy_tues_6_files/figure-latex/unnamed-chunk-4-1.pdf}

\begin{Shaded}
\begin{Highlighting}[]
\DocumentationTok{\#\#\#\#\# Ulta Plot \#\#\#\#\#}

\NormalTok{ulta\_plot }\OtherTok{\textless{}{-}}\NormalTok{ ulta\_data }\SpecialCharTok{\%\textgreater{}\%}
  \FunctionTok{ggplot}\NormalTok{(}\FunctionTok{aes}\NormalTok{(}\AttributeTok{x =}\NormalTok{ numbers,             }
             \AttributeTok{y =}\NormalTok{ lightness,}
             \AttributeTok{color =}\NormalTok{ lightness))}\SpecialCharTok{+}               \DocumentationTok{\#\#\# What I chose for my x/y axis}
  \FunctionTok{geom\_point}\NormalTok{()}\SpecialCharTok{+}                                 \DocumentationTok{\#\#\# Type of plot I used}
  \FunctionTok{facet\_wrap}\NormalTok{(}\SpecialCharTok{\textasciitilde{}}\NormalTok{brand, }\AttributeTok{scales =} \StringTok{"free"}\NormalTok{)}\SpecialCharTok{+}          \DocumentationTok{\#\#\# How I got each brand to have its own small plot}
  \FunctionTok{theme\_classic}\NormalTok{()}\SpecialCharTok{+}                              \DocumentationTok{\#\#\# Theme }
  \FunctionTok{labs}\NormalTok{(}\AttributeTok{x =} \StringTok{"Foundation Number"}\NormalTok{,}
       \AttributeTok{y =} \StringTok{"Shade Range"}\NormalTok{,}
       \AttributeTok{subtitle =} \StringTok{"This plot shows the shade ranges of foundation }
\StringTok{avaliable at Ulta based on the brand."}\NormalTok{)}\SpecialCharTok{+}        \DocumentationTok{\#\#\# new labels for the plot \#\#}
  \FunctionTok{theme}\NormalTok{(}\AttributeTok{legend.box =} \StringTok{"none"}\NormalTok{,}
        \AttributeTok{axis.title =} \FunctionTok{element\_text}\NormalTok{(}\AttributeTok{size =} \DecValTok{10}\NormalTok{),}
        \AttributeTok{title =} \FunctionTok{element\_text}\NormalTok{(}\AttributeTok{size =} \DecValTok{16}\NormalTok{),}
        \AttributeTok{plot.subtitle =} \FunctionTok{element\_text}\NormalTok{(}\AttributeTok{size =} \DecValTok{11}\NormalTok{),}
        \AttributeTok{legend.position =} \StringTok{"none"}\NormalTok{)}\SpecialCharTok{+}             \DocumentationTok{\#\#\# How I changed certain aspects of the plot}
  \FunctionTok{ggsave}\NormalTok{(}\FunctionTok{here}\NormalTok{(}\StringTok{"tidy\_tuesday\_6"}\NormalTok{, }\StringTok{"output"}\NormalTok{, }\StringTok{"ulta\_plot.png"}\NormalTok{),}
       \AttributeTok{width =} \DecValTok{10}\NormalTok{, }\AttributeTok{height =} \DecValTok{7}\NormalTok{)}

\NormalTok{ulta\_plot}
\end{Highlighting}
\end{Shaded}

\includegraphics{tidy_tues_6_files/figure-latex/unnamed-chunk-4-2.pdf}

\hypertarget{putting-the-plots-together}{%
\subsubsection{Putting the plots
together}\label{putting-the-plots-together}}

\begin{Shaded}
\begin{Highlighting}[]
\NormalTok{seph\_plot}\SpecialCharTok{+}\NormalTok{ulta\_plot }\SpecialCharTok{+}
  \FunctionTok{plot\_layout}\NormalTok{(}\AttributeTok{guides =} \StringTok{\textquotesingle{}collect\textquotesingle{}}\NormalTok{)}\SpecialCharTok{+}
\FunctionTok{ggsave}\NormalTok{(}\FunctionTok{here}\NormalTok{(}\StringTok{"tidy\_tuesday\_6"}\NormalTok{, }\StringTok{"output"}\NormalTok{, }\StringTok{"seph\_ulta\_plot.png"}\NormalTok{),}
       \AttributeTok{width =} \DecValTok{10}\NormalTok{, }\AttributeTok{height =} \DecValTok{7}\NormalTok{)}
\end{Highlighting}
\end{Shaded}

\includegraphics{tidy_tues_6_files/figure-latex/unnamed-chunk-5-1.pdf}

\end{document}
